\section*{Kurzfassung}
\begin{otherlanguage}{german}
Mit der Entwicklung von Freie-Elektronen-Lasern im Röntgen Wellenlängenbereich, wie der \textit{Linac Coherent Light Source} (LCLS), sind strukturelle Studien an Nanopartikeln möglich. Eine Herausforderung bei diesen Studien mit höchst intensiven Röntgenpulsen ist die Schädigung der Nanopartikel, da dies die Auflösung limitiert. Nur ein fundiertes Verständnis von Strahlungsschäden, wie die Nanoplasma-Bildung und -Expansion, kann die Auflösung verbessern. Weiterhin wurde vorgeschlagen, dass eine Schutzhülle die schädigenden Prozesse und die Expansion von Nanopartikeln verlangsamen können.\\[0.6\baselineskip]
%
Diese Arbeit untersucht die Nanoplasma-Bildung und -Expansion in nanometergroßen He-, Xe- und HeXe-Cluster, wobei HeXe-Cluster ein Modellsystem sind, um die Funktion von solchen Schutzhüllen zu untersuchen. Die Nanopartikel werden durch eine Überschall-Expansion von kaltem Gas ins Vakuum gebildet und werden an dem Röntgenlaser LCLS mit Röntgenstrahlungs-Doppelpulsen zeitaufgelöst untersucht. Hierbei startet ein Pump-Puls die Nanoplasma-Entwicklung mit Intensitäten von \SI{\sim 2e16}{\watt\per\square\centi\meter} und ein Probe-Puls erzeugt Beugungsbilder von dem Nanoplasma zu einem späteren Zeitpunkt $\Delta t$, mit Intensitäten von \SI{\sim 2e17}{\watt\per\square\centi\meter}. Die Verzögerung $\Delta t$ wird zwischen \SIrange{0}{800}{\femto\second} variiert und die finale Ionenverteilung wird mit einem Flugzeitmassenspektrometer (TOF) gemessen. Das Experiment dieser Arbeit ist in der LAMP Kammer des AMO Instruments an der LCLS durchgeführt worden. Ein wesentlicher experimenteller Teil dieser Arbeit war der Bau, die Inbetriebnahme, und die Teilnahme an Design-Diskussionen von der LAMP Kammer.\\[0.6\baselineskip]
%
Die Studie zeigt Einzelschuss Beugungsbilder und TOF Spektren von Xe-, He- und HeXe-Clustern mit einer bisher unerreichten Auflösung. So ist beispielsweise die Rekonstruktion eines \SI{\sim25}{\nano\meter} Radius großen Xe-Cluster mit \SI{\sim 6}{\nano\meter} Auflösung gezeigt. Die Beugungsbilder und Rekonstruktionen zeigen, wie sich die Nanopartikel zu einem expandierenden Nanoplasma entwickeln. Die Auswertung einiger hundert Einzelschussbeugungsbilder zeigt die Ausdehung der Xe-Cluster um etwa \SI{20}{\percent} binnen \SI{800}{\femto\second}. 
%Eine Auswertung einiger hundert Einzelschussbeugungsbilder zeigt, dass der ursprüngliche Xe-Clusterradius um etwa \SI{20}{\percent} im Verlauf von \SI{800}{\femto\second} steigt. Die Expansionsgeschwindigkeit von einem Xe-cluster beträgt \SI{\sim 15250}{\meter\per\second} und die Elektronen Temperatur ist \SI{\sim125}{\electronvolt}.
Rekonstruktionen und 2D Computer Simulationen belegen, dass HeXe-Cluster eine Rosinenkuchenstruktur besitzen, da Xe-Atome in dem He-Tröpfchen zu mehreren kleinen Clustern kondensieren. TOF Spektren zeigen, dass Xenon kinetische Energie an die Schutzhülle aus Helium transferiert und, dass die Schutzhülle als Elektronenreservoir für das Xenon dient. Die Beugungsbilder der HeXe-Cluster weisen auf, dass Xe-Cluster in den ersten \SI{800}{\femto\second} praktisch nicht expandieren, während die He-Schutzhülle einer Nanoplasma-Expansion unterliegt. Abschließend wird die Nanoplasma-Expansion von reinen Xe- und He-Clustern, sowie gemischten HeXe-Clustern miteinander verglichen.
\end{otherlanguage}