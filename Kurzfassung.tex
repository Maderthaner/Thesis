\section*{Kurzfassung}
\begin{otherlanguage}{german}
Mit der Entwicklung von Freie-Elektronen-Lasern im Röntgen Wellenlängenbereich, wie der \textit{Linac Coherent Light Source} (LCLS), sind strukturelle Studien an Nanopartikeln möglich. Eine Herausforderung bei diesen Studien mit höchst intensiven Röntgenpulsen ist die Beschädigung der Nanopartikel, da dies die Auflösung limitiert. Nur ein fundiertes Verständnis der Strahlungsschadenprozesse, wie die Nanoplasma Expansion, kann die Auflösung verbessern. Weiterhin wurde vorgeschlagen, dass eine Schutzhülle die Nanoplasma-Expansion von Nanopartikeln verlangsamen können.\\[0.6\baselineskip]
%
Diese Arbeit untersucht die Nanoplasma-Transformation und -Expansion in nanometergroßen He-, Xe- und HeXe-cluster, wobei HeXe-cluster ein Modellsystem sind, um den Nutzen von solchen Schutzhüllen zu untersuchen. Die Nanopartikel werden durch eine Überschall Expansion von Gasen ins Vakuum gebildet und werden an der LCLS mit Röntgenstrahlungs-Doppelpulsen zeitaufgelöst untersucht. Hierbei startet ein Pump-Puls die Nanoplasma-Entwicklung mit Intensitäten von \SI{\sim 2e16}{\watt\per\square\centi\meter} und ein Probe-Puls erzeugt Beugungsbilder von dem Nanoplasma zu einem späteren Zeitpunkt $\Delta t$, mit Intensitäten von \SI{\sim 2e17}{\watt\per\square\centi\meter}. $\Delta t$ wird zwischen \SIrange{0}{800}{\femto\second} variiert und die finale Ionenverteilung wird mit einem Flugzeitmassenspektrometer (TOF) gemessen. Das Experiment dieser Arbeit ist in der LAMP Kammer, des AMO Instruments durchgeführt worden. Ein wesentlicher Teil dieser Arbeit war der Bau, die Inbetriebnahme, und die Teilnahme an Design-Diskussionen von der LAMP Kammer.\\[0.6\baselineskip]
%
Die Studie zeigt Einzelschuss Beugungsbilder und TOF Spektren von Xe-, He- und HeXe-clustern mit einer bisher unerreichten Auflösung. So ist beispielsweise die Rekonstruktion eines \SI{\sim25}{\nano\meter} Radius großen Xe-cluster mit \SI{\sim 6}{\nano\meter} Auflösung gezeigt. Die Beugungsbilder und Rekonstruktionen zeigen, wie sich die Nanopartikel zu einem expandierenden Nanoplasma entwickeln. Die Auswertung einiger hundert Einzelschussbeugungsbilder zeigt die Ausdehung der Xe-cluster um etwa \SI{20}{\percent} binnen \SI{800}{\femto\second}. 
%Eine Auswertung einiger hundert Einzelschussbeugungsbilder zeigt, dass der ursprüngliche Xe-Clusterradius um etwa \SI{20}{\percent} im Verlauf von \SI{800}{\femto\second} steigt. Die Expansionsgeschwindigkeit von einem Xe-cluster beträgt \SI{\sim 15250}{\meter\per\second} und die Elektronen Temperatur ist \SI{\sim125}{\electronvolt}.
Rekonstruktionen und 2D Computer Simulationen zeigen, dass sich HeXe-cluster ähnlich eines Rosinenkuchenmodells anordnen, da Xe-Atome in dem He-Tröpfchen zu mehreren kleinen Clustern kondensieren. TOF Spektren zeigen, dass Xe-Cluster kinetische Energie an die Schutzhülle aus Helium transferieren und, dass die Schutzhülle als Elektronenreservoir für die Xe-cluster dient. Die Beugungsbilder der HeXe-cluster zeigen, dass Xe-cluster in den ersten \SI{800}{\femto\second} nicht messbar expandieren, während die He-Schutzhülle einer Nanoplasma-Expansion unterliegt. Abschließend wird die Nanoplasma-Expansion von reinen Xe- und He-Clustern, sowie gemischten HeXe-clustern miteinander verglichen.
\end{otherlanguage}