\section*{Kurzfassung}
\begin{otherlanguage}{german}
Mit der Entwicklung von \textit{Freien Elektronen Lasern} im Röntgen Wellenlängenbereich, wie die Linac Coherent Light Source (LCLS), sind strukturelle Studien an Nanopartikeln bald möglich. Eine Herausforderung bei diesen Studien ist die beschädigung der Nanopartikel durch höchst intensive Röntgenpulse, die die Auflösung limitiert. Nur ein fundiertes Verständnis der zu grundeliegenden Prozesse, wie die \textit{Nanoplasma} Entstehung, kann die Auflösung verbessern. Vor kurzem wurde vorgeschlagen, dass \textit{Opferschichten} die Nanoplasma Expansion von Nanopartikeln verlangsamen können.\\[0.4\baselineskip]
%
Diese Arbeit untersucht die Nanoplasma Entwicklung und Expansion in nanometer großen He-, Xe- und HeXe-cluster, wobei HeXe-cluster ein Modellsystem für Opferschichten sind. Die Nanopartikel werden durch eine überschall Gas Expansion ins Vakuum formiert und mit einer ultraschnellen Röntgenstrahlungs Pump--Röntgenstrahlungs Probe Methode von der LCLS untersucht. Hierbei startet der Pump-Puls die Nanoplasma Entwicklung mit Intensitäten von \SI{\sim e17}{\watt\per\square\centi\meter} und der Probe-Puls erzeugt Beugungsbilder von dem Nanoplasma zu einem späteren Zeitpunkt, $\Delta t$, mit Intensitäten von \SI{\sim e18}{\watt\per\square\centi\meter}. $\Delta t$ wird zwischen \SIrange{0}{800}{\femto\second} variiert und die finale Ionen Komposition wird mit einem Flugzeitmassenspektrometer (TOF) gemessen. Das Experiment dieser Arbeit ist in der LAMP Kammer, des AMO Instruments durchgeführt worden. Ein wesentlicher Teil dieser Arbeit war der Bau, die Inbetriebnahme, und die Teilnahme an Design-Diskussionen von der LAMP Kammer.\\[0.4\baselineskip]
%
Die Studie zeigt, Einzelschuss Beugungsbilder und TOF Spektren von Xe-, He- und HeXe-cluster mit einer bisher unerreichten Auflösung, zum Beispiel ist die Rekonstruktion eines \SI{\sim25}{\nano\meter} radius großen Xe-cluster mit \SI{\sim 6}{\nano\meter} Auflösung gezeigt. Die Beugungsbilder und Rekonstruktionen zeigen, wie sich die Nanopartikel zu einem expandierenden Nanoplasma entwickeln. Eine Auswertung einiger hundert Einzelschuss Beugungsbilder zeigt das der initiale Xe-Clusterradius um etwa \SI{20}{\percent} über \SI{800}{\femto\second} steigt. Die Expansionsgeschwindigkeit von einem Xe-cluster beträgt \SI{\sim 15250}{\meter\per\second} und die Elektronen Temperatur ist \SI{\sim125}{\electronvolt}. Rekonstruktionen und 2D Computer Simulationen zeigen, dass sich HeXe-cluster ähnlich eines \textit{Rosinenkuchenmodells} anordnen, indem Xe-Atome in dem He-Tröpfchen zu mehreren kleinen Clustern kondensieren. TOF Spektren zeigen, dass Xe-Cluster kinetische Energie an die Helium Opferschicht transferieren und das die Opferschicht als Elektronenreservoir für die Xe-cluster dient. Die Beugungsbilder der HeXe-cluster zeigen, dass Xe-cluster in den ersten \SI{800}{\femto\second} nicht expandieren während die He-Opferschicht einer Nanoplasma Expansion unterliegt. Ebenfalls wird die Nanoplasma Expansion von reinen Xe- und He-Clustern, sowie gemischten HeXe-clustern miteinander verlgichen.
\end{otherlanguage}