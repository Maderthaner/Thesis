\chapter{Summary and outlook}\label{ch:summary_outlook}
For the present study, a novel, accelerator-based X-ray pump--X-ray probe technique was pioneered \citep{Lutman-2013-PRL} and the new soft X-ray end-station LAMP was built and commissioned \citep{Ferguson-2015-JSR}. Using this new experimental setup, a coincident single particle imaging (SPI) and time-of-flight mass spectroscopy method was employed. Diffraction images of single, nanometer-sized aerosol particles were measured with thus-far unprecedented resolution. The successful combination of data from multiple detectors increased the ``numerical aperture'' and extended the dynamic range of the detector system. The developed coincident imaging and spectroscopy, X-ray pump--X-ray probe technique allowed the study of X-ray-induced dynamics on the few-ten-femtosecond timescale with a spatial resolution in diffractive imaging on the few nanometer lengthscale.\\[1\baselineskip]
%
The work experimentally investigates the nanoplasma transition in rare-gas clusters. Single, nanometer-sized, rare-gas clusters, namely superfluid He-, solid Xe- and mixed HeXe-cluster are ``pumped'' with an X-ray pulse to undergo the phase transition to a nanoplasma. This transition is nothing different but the radiation damage that a typical biological sample receives due to the interaction with highly intense X-ray pulses. This transition is ``probed'' with another X-ray pulse at a later time delay. The here studied delay range is $\Delta t=$ \SIrange{0}{800}{\femto\second}. The coincident imaging and spectroscopy measuring method reveals that pristine He- and Xe-clusters become highly ionized and form a nanoplasma. The nanoplasma from Xe- and He-clusters exhibits stark radiation damage due to an expansion of the plasma. Xe-clusters, for example, undergo a radial expansion that shows in diffraction images at the first delay step at $\Delta t = 120$ fs. Over the course of 800 fs, the radial expansion manifests in a \SI{\sim 20}{\percent} increase of their initial average radius, which is $r\approx 61$ nm. The hot Xe-nanoplasma is heated to electron temperatures of \SI{\sim 125}{\electronvolt}. This thesis then shows that mixed HeXe-clusters, generated through the pickup-principle at doping levels up to \SI{0.5}{\percent}\footnote{\SI{0.5}{\percent} as many xenon atoms as helium atoms.}, arrange in a \textit{plum-pudding}\index{plum-pudding} configuration. In a plum-pudding cluster structure, multiple Xe-clusters condense within a larger He-droplet at different locations. When these mixed HeXe-clusters are pumped with intense X-rays from LCLS, Xe-clusters are the main absorbent of the radiation. But, data show that Xe-clusters are barely ionized and, at current resolution, appear to be undamaged inside the He-droplet. The weak absorbing helium experiences stark ionization and the He-droplet shows signs of radiation damage. This suggests that the He-droplet functions as a sacrificial shell around the Xe-particles. Two processes are identified that protect the Xe-clusters within the He-droplet; one, a kinetic energy transport from the Xe-clusters to the He-droplet; two, the He-droplet supplies electrons to the initially ionized Xe-particles, minimizing their ionization level.\\[1\baselineskip]
%
%In more detail. Single Xe-cluster were illuminated by the X-ray pump beam from LCLS to induce the nanoplasma transition and at a later time delay probed by ultrafast pulses from LCLS to create a snapshot of the transition. The study of diffraction pattern shows an expansion of the cluster's electron density of XXX \% raidus over 800 fs. This is comparable to other studies using optical pump--probe methods \citep{Gorkhover-2016-NatPho}. As the relaxation processes in larger atoms are on the few femtosecond timescale, the total absorbed energy is the key driver of the nanoplasma expansion in larger clusters. As the cluster become smaller, the X-ray pump--X-ray probe technique reveals a to optical methods hidden resonance in Xe-atoms and small cluster ($\sim <5 nm$ radius) that may be attributed to relaxation pathways only accessed after core-electron ionization \citep{Ho-2014-PRL}.\\
%To prevent the nanoplasma transition from damaging the sample object (Xe-cluster), a helium layer is placed around the sample. These heterogeneous cluster consisting of a He-shell and a Xe-core are investigated using the same methods. The study shows that multiple Xe-cluster arrange in a plum-pudding type arrangement. The analysis of diffraction pattern shows that the He-droplet undergoes the nanoplasma transition, thus exhibits radiation damage, while the Xe-cluster within the droplet show no radiation damage over a pump--probe delay of 800 fs. Simultaneously, the ion-spectroscopy shows a significant increase in helium high-charge states and momentum compared to pristine He-droplets. Additionally, only few Xe-ions are detected in the final state snapshot of the time-of-flight spectrometer. The sacrificial tampered He-layer thus supplies electrons to the Xe-ions and transports kinetic energy away from the sample.\\
%
The results of this work are beneficial to a variety of fields. First, the work from \citep{Hoener-2008-JPB,Gorkhover-2016-NatPho} are expanded to include reconstructions of nanoparticles and a time-resolved X-ray pump--X-ray probe study. This study shows that a tamper layer can be used to inhibit effects of radiation damage in the diffraction pattern. This may be interesting for the SPI community, where it is foreseen that radiation damage will be a limiting factor in the ultimate achievable resolution \citep{Aquila-2015-StrucDyn}. Second, new and existing SPI data may make use of combining multiple detectors thereby increasing the resolution by a factor \num{\sim 5}. Combined diffraction images unfold their full potential when combined with an EMC algorithm \citep{Loh-2009-PRE}. This allows an orientation and averaging of diffraction images, thus enabling a high-resolution 3D reconstruction of the sample while making use of the asymmetric detector geometry. Finally, the pioneered X-ray pump--X-ray probe technique can be and already has been extended to various other atomic and molecular physics experiments such as \cite{Lehmann-2016-PRA,Kimberg-2016-FD,Al-Haddad-2017-unpublished,Ferguson-2016-SciAdv,Picon-2016-NatComm}, allowing entirely new insights in previously unreachable regimes.
%
%
%