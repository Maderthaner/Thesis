\chapter{Summary and outlook}\label{ch:summary_outlook}
For the present study, a novel, accelerator based X-ray pump--X-ray probe technique was pioneered \citep{Lutman-2013-PRL} and the new soft X-ray end-station LAMP was built and commissioned \citep{Ferguson-2015-JSR}. Using this new experimental setup, a coincident single particle imaging (SPI) and time-of-flight mass spectroscopy method was employed. Diffraction images of single nanometer sized aerosol particles were measured with thus far unprecedented resolution. The successful combination of data from multiple detectors increased the numeric aperture and extended the dynamic range of the detector system. The developed coincident imaging and spectroscopy, X-ray pump--X-ray probe technique allowed the study of X-ray induced dynamics on the few ten femtosecond timescale with a spatial resolution in diffractive imaging on the few nanometer lengthscale.\\
%
The present work experimentally investigates the nanoplasma transition in rare-gas cluster. Therefore, single, nanometer-sized, rare-gas cluster, namely superfluid He-, solid Xe- and mixed HeXe-cluster are pumped with X-rays to undergo the phase transition to a nanoplasma, i.e., experience radiation damage. This transition is probed at a later time delay $\Delta t=0$ to $800$ fs. The coincident imaging and spectroscopy measuring method reveals that pristine He- and Xe-cluster become highly ionized and form a nanoplasma. The nanoplasma from Xe- and He-cluster exhibits stark structural damage due to the expansion of the plasma. Xe-clusters, for example, exhibit a radial expansion, i.e. structural damage, that shows in the first delay step at $\Delta t = 120$ fs and over the course of 800 fs manifests in a $\sim 20 \%$ increase of their initial average radius $r\approx 61$ nm. The hot Xe-nanoplasma is heated to electron temperatures of $\sim 125$ eV. This thesis then shows that mixed HeXe-cluster, generated through the pickup-principle at doping levels up to $0.5\%$\footnote{$0.5\%$ as many xenon atoms as helium atoms.}, arrange in a \textit{plum-pudding}\index{plum-pudding} type. In a pump-pudding cluster structure, multiple Xe-cluster condensate in a greater He-droplet at different locations. When these mixed HeXe-cluster are pumped with intense X-rays from LCLS, Xe-cluster are the main absorbent of the radiation. Data show that Xe-cluster are barely ionized and, at current resolution, exhibit no structural damage in the He-droplet. The weak absorbent helium experiences stark ionization and the He-droplet shows structural damage. Data suggests that the He-droplet functions as sacrificial shell around the Xe-particles due to two processes. One, a kinetic energy transport from the Xe-clusters to the He-droplet, and two, the He-droplet supplies electrons to the initially ionized Xe-particles, minimizing their electronic damage.\\
%
%In more detail. Single Xe-cluster were illuminated by the X-ray pump beam from LCLS to induce the nanoplasma transition and at a later time delay probed by ultrafast pulses from LCLS to create a snapshot of the transition. The study of diffraction pattern shows an expansion of the cluster's electron density of XXX \% raidus over 800 fs. This is comparable to other studies using optical pump--probe methods \citep{Gorkhover-2016-NatPho}. As the relaxation processes in larger atoms are on the few femtosecond timescale, the total absorbed energy is the key driver of the nanoplasma expansion in larger clusters. As the cluster become smaller, the X-ray pump--X-ray probe technique reveals a to optical methods hidden resonance in Xe-atoms and small cluster ($\sim <5 nm$ radius) that may be attributed to relaxation pathways only accessed after core-electron ionization \citep{Ho-2014-PRL}.\\
%To prevent the nanoplasma transition from damaging the sample object (Xe-cluster), a helium layer is placed around the sample. These heterogeneous cluster consisting of a He-shell and a Xe-core are investigated using the same methods. The study shows that multiple Xe-cluster arrange in a plum-pudding type arrangement. The analysis of diffraction pattern shows that the He-droplet undergoes the nanoplasma transition, thus exhibits radiation damage, while the Xe-cluster within the droplet show no radiation damage over a pump--probe delay of 800 fs. Simultaneously, the ion-spectroscopy shows a significant increase in helium high-charge states and momentum compared to pristine He-droplets. Additionally, only few Xe-ions are detected in the final state snapshot of the time-of-flight spectrometer. The sacrificial tampered He-layer thus supplies electrons to the Xe-ions and transports kinetic energy away from the sample.\\
%
These results of this work are beneficial to a variety of fields. First, the insights gained from \citep{Hoener-2008-JPB,Gorkhover-2016-NatPho} are extended and it is shown that tampered layer can be used to inhibit effects of radiation damage in the diffraction pattern. This may be interesting for the SPI community, where it is foreseen that radiation damage will be a limiting factor in the ultimate achievable resolution \citep{Aquila-2015-StrucDyn}. Second, new and existing SPI data may make use of combining multiple detectors thereby increasing the resolution by a factor $\sim 5$. Combined diffraction images unfold their full potential when combined with an EMC algorithm \citep{Loh-2009-PRE}. This allows an orientation and averaging of diffraction images thus enabling a high-resolution 3D reconstruction of the sample, while making use of the asymmetric detector geometry. Finally, the pioneered X-ray pump--X-ray probe technique can be and already has been extended to various other atomic and molecular physics experiments such as \citep{Picon-2016-NatComm,Lehmann-2016-PRA,Kimberg-2016-FD,Al-Haddad-2017-unpublished,Ferguson-2016-SciAdv}, allowing entire new insights in thus far unreachable regimes.
%
%
%