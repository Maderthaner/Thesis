\chapter{Summary and outlook}\label{ch:summary_outlook}
%
%This work experimentally investigates the nanoplasma formation and expansion in rare-gas clusters. The nanoplasma formation gathers the underlying effects that a typical biological molecule exhibits upon interaction with highly intense X-ray pulses and is often referred to as sample damage \cite{Neutze-2000-Nature,Arbeiter-2011-NJP}. Sacrificial layers are thought to slow the nanoplasma expansion \cite{Hau-Riege-2007-PRL,Hau-Riege-2010-PRL} and recently HeXe-clusters were proposed to study the sacrificial layers \cite{Mikaberidze-2008-PRA}.\\[0.9\baselineskip]
This work experimentally investigates the ionization and expansion dynamics of nanometer-sized samples in intense X-ray pulses using a novel X-ray pump--X-ray probe technique. During intense X-ray pulses from free-electron lasers (FELs) all nanometer-sized samples are transformed into a nanoplasma. The nanoplasma transformation consists of a rapid ionization and expansion which is often referred to as sample damage \cite{Neutze-2000-Nature,Arbeiter-2011-NJP}. Sacrificial layers around the sample have been suggested as method to slow the nanoplasma expansion \cite{Hau-Riege-2007-PRL,Hau-Riege-2010-PRL}. To investigate the effects of sacrificial layers on the nanoplasma formation of a sample, we use pristine xenon clusters and xenon clusters embedded in a helium droplet. Our work shows that the xenon clusters inside helium droplets show a greatly slowed X-ray induced damage processes.\\[1\baselineskip]
%Here, HeXe-clusters reach Xe-doping levels of up to \SI{\sim 0.6}{\percent}\footnote{This means \SI{\sim 0.6}{\percent} as many xenon atoms as helium atoms.}.
For the experiments, the nanometer-sized superfluid He-droplets and solid Xe-clusters are created through a supersonic gas expansion into the vacuum. The heterogeneous HeXe-clusters are created through the pickup concept. To study the nanoplasma transformation of these samples, a novel, accelerator-based X-ray pump--X-ray probe technique is employed at the Linear Coherent Light Source \cite{Lutman-2013-PRL}. This technique produces two pulses with a pump-pulse on the order of \SI{\sim 2e16}{\watt\per\square\centi\meter} to ignite the nanoplasma transformation and a factor 10 stronger probe-pulse to image the nanoplasma transformation. The pump--probe pulses are delayed enabling a time-resolved study of the nanoplasma transformation in the thus far unreached range of \SIrange{0}{800}{\femto\second}. The new soft X-ray imaging end-station LAMP was partly designed, built, and commissioned as part of this thesis \citep{Ferguson-2015-JSR}. Using this new experimental setup, a coincident single particle imaging and time-of-flight mass spectroscopy method is employed \cite{Gorkhover-2012-PRL}. Data from multiple imaging detectors are merged, which virtually increases the detector area and virtually extends the dynamic range of the detector system. This allows the measurement of diffraction images with thus far unprecedented resolution. The combination of these experimental methods allows the study of X-ray-induced dynamics on the femtosecond timescale with a spatial resolution in diffractive imaging on the few nanometer lengthscale.\\[1\baselineskip]
%For the present study, a novel, accelerator-based X-ray pump--X-ray probe technique was pioneered \citep{Lutman-2013-PRL} and the new soft X-ray end-station LAMP was partly designed, built, and commissioned \citep{Ferguson-2015-JSR} at the Linear Coherent Light Source. Using this new experimental setup, a coincident single particle imaging and time-of-flight mass spectroscopy method is employed. Data from multiple imaging detectors are combined, which virtually increases the detector area and virtually extends the dynamic range of the detector system. This allows the measurement of diffraction images with thus far unprecedented resolution. The combination of these experimental methods allows the study of X-ray-induced dynamics on the femtosecond timescale with a spatial resolution in diffractive imaging on the few nanometer lengthscale.\\[0.9\baselineskip]
%
%In this thesis experiment, nanometer-sized superfluid He-, solid Xe-clusters are created through a supersonic gas expansion into the vacuum and heterogeneous HeXe-clusters are created through the pickup concept. HeXe-clusters reach Xe-doping levels of up to \SI{\sim 0.6}{\percent}\footnote{This means \SI{\sim 0.6}{\percent} as many xenon atoms as helium atoms.}. Rare-gas clusters are an ideal sample as they can be easily created, are variable in size, and do not dissipate energy in surrounding media. These clusters are then investigated using an XFEL-based X-ray pump--X-ray probe technique. Here, the pump-pulse initiates a nanoplasma formation of a single cluster, and at a later time, $\Delta t$, a probe-pulse creates a diffraction image of the nanoparticle. The studied delay range is $\Delta t=$ \SIrange{0}{800}{\femto\second}, which extends the observation-window of X-ray induced nanoplasma formation by over \SI{700}{\femto\second}. The diffraction images from multiple detectors are combined and phase-retrievals are performed to analyze the data in reciprocal- and real-space.\\[0.8\baselineskip]
%
The experimental data shows that pristine Xe- and He-clusters exhibit a surface-softening in which first the outer atomic layers of the nanoparticle are ``exploding'' off the sample and inner-layers follow. Xe-clusters undergo a radial expansion that shows in diffraction images within the first delay step at $\Delta t =$ \SI{120}{\femto\second}. Over the course of \SI{800}{\femto\second}, the radial expansion manifests in a \SI{\sim 20}{\percent} increase of their initial average radius, which is $r\approx$ \SI{61}{\nano\meter}. The Xe-nanoplasma is heated to electron temperatures of \SI{\sim 125}{\electronvolt}. He-droplets exhibits similar but less violent sample damage. Heterogeneous HeXe-clusters likely agglomerate in a plum-pudding\index{plum-pudding} configuration. In this plum-pudding HeXe-cluster structure, multiple Xe-clusters condense within a larger He-droplet at different locations. When these mixed HeXe-clusters are pumped with intense X-rays from LCLS, Xe-clusters are dominantly absorbing the radiation. But, data show that Xe-clusters are barely ionized and, at current resolution, appear to be undamaged inside the He-droplet. The weak absorbing helium experiences stark ionization and the He-droplet shows signs of a surface-softening. This indicates that the He-droplet functions as a sacrificial shell around the Xe-particles, which slows their surface-softening. However, X-ray induced damage that becomes detectable at higher resolutions may still be possible.\\[0.5\baselineskip]
%
%This thesis finds that Xe- and He-clusters exhibit a surface-softening in which the outer atomic layers of the nanoparticle are ``exploding'' off the sample first and inner-layers follow. Xe-clusters, for example, undergo a radial expansion that shows in diffraction images within the first delay step at $\Delta t =$ \SI{120}{\femto\second}. Over the course of \SI{800}{\femto\second}, the radial expansion manifests in a \SI{\sim 20}{\percent} increase of their initial average radius, which is $r\approx$ \SI{61}{\nano\meter}. The Xe-nanoplasma is heated to electron temperatures of \SI{\sim 125}{\electronvolt}. Heterogeneous HeXe-clusters likely agglomerate in a plum-pudding\index{plum-pudding} configuration. In this plum-pudding HeXe-cluster structure, multiple Xe-clusters condense within a larger He-droplet at different locations. When these mixed HeXe-clusters are pumped with intense X-rays from LCLS, Xe-clusters are dominantly absorbing the radiation. But, data show that Xe-clusters are barely ionized and, at current resolution, appear to be undamaged inside the He-droplet. The weak absorbing helium experiences stark ionization and the He-droplet shows signs of a surface-softening. This indicates that the He-droplet functions as a sacrificial shell around the Xe-particles and slows their surface-softening. However, X-ray induced damage that becomes detectable at higher resolutions may still be possible. Two processes are identified that could slow the nanoplasma formation of the Xe-clusters within the He-droplet: one, a kinetic energy transport from the Xe-clusters to the He-droplet; two, the He-droplet supplies electrons to the initially ionized Xe-particles, minimizing their ionization level.\\[0.5\baselineskip]
%
%In more detail. Single Xe-cluster were illuminated by the X-ray pump beam from LCLS to induce the nanoplasma transition and at a later time delay probed by ultrafast pulses from LCLS to create a snapshot of the transition. The study of diffraction pattern shows an expansion of the cluster's electron density of XXX \% raidus over 800 fs. This is comparable to other studies using optical pump--probe methods \citep{Gorkhover-2016-NatPho}. As the relaxation processes in larger atoms are on the few femtosecond timescale, the total absorbed energy is the key driver of the nanoplasma expansion in larger clusters. As the cluster become smaller, the X-ray pump--X-ray probe technique reveals a to optical methods hidden resonance in Xe-atoms and small cluster ($\sim <5 nm$ radius) that may be attributed to relaxation pathways only accessed after core-electron ionization \citep{Ho-2014-PRL}.\\
%To prevent the nanoplasma transition from damaging the sample object (Xe-cluster), a helium layer is placed around the sample. These heterogeneous cluster consisting of a He-shell and a Xe-core are investigated using the same methods. The study shows that multiple Xe-cluster arrange in a plum-pudding type arrangement. The analysis of diffraction pattern shows that the He-droplet undergoes the nanoplasma transition, thus exhibits radiation damage, while the Xe-cluster within the droplet show no radiation damage over a pump--probe delay of 800 fs. Simultaneously, the ion-spectroscopy shows a significant increase in helium high-charge states and momentum compared to pristine He-droplets. Additionally, only few Xe-ions are detected in the final state snapshot of the time-of-flight spectrometer. The sacrificial tampered He-layer thus supplies electrons to the Xe-ions and transports kinetic energy away from the sample.\\
%
The results of this work are beneficial to a variety of fields. First, the works from References \citep{Hoener-2008-JPB,Gorkhover-2016-NatPho} are expanded to include reconstructions of nanoparticles and a time-resolved X-ray pump--X-ray probe technique to investigate the nanoplasma formation. Second, it was found through diffractive imaging that HeXe-cluster agglomerate in a plum-pudding structure. Such systems could be a useful model for other heterogeneous clusters. Third, the time-resolved data of HeXe-clusters indicates that the nanoplasma expansion of Xe-clusters is slowed within the He-droplet. Therefore, helium could be used as sacrificial layer material to slow the sample damage progression as such He-tamper sample injection methods are currently discussed \cite{Bielecki-2016-PC}. This may be interesting for the single-particle imaging community, where it is foreseen that radiation damage will be a limiting factor in the ultimate achievable resolution \citep{Aquila-2015-StrucDyn}. Additionally, the imaging community will be able to take advantage of the improved detector geometry and merged images described in this thesis. This method increases the previously obtained resolution in single-shot diffractive imaging by a factor \num{\sim 5} and virtually increases the dynamic range. The method will unfold its full potential when asymmetric diffraction images are combined with an EMC algorithm \citep{Loh-2009-PRE}. The algorithm allows an orientation and averaging of the asymmetric diffraction images enabling sub nanometer-resolution 3D reconstructions. Finally, the X-ray pump--X-ray probe approach pioneered during this thesis experiment has already been extended to various other experiments, such as References \cite{Lehmann-2016-PRA,Kimberg-2016-FD,Al-Haddad-2017-unpublished,Ferguson-2016-SciAdv,Picon-2016-NatComm}, allowing entirely new insights previously unreachable.
%
%The results of this work are beneficial to a variety of fields. First, the work from Reference \citep{Hoener-2008-JPB,Gorkhover-2016-NatPho} are expanded to include reconstructions of nanoparticles and a time-resolved X-ray pump--X-ray probe method to investigate the nanoplasma formation. Second, HeXe-cluster likely agglomerate in a plum-pudding structure and it could be a useful model for other heterogeneous clusters. Third, the time-resolved data of HeXe-clusters indicates that the nanoplasma expansion of Xe-clusters is slowed within the He-droplet. This data indicates that helium could be used as sacrificial layer material and indeed slow the sample damage progression. This may be interesting for the SPI community, where it is foreseen that radiation damage will be a limiting factor in the ultimate achievable resolution \citep{Aquila-2015-StrucDyn} and sample injection methods including He-layers are discussed. Second, new and existing SPI data may make use of combining multiple detectors thereby increasing the resolution by a factor \num{\sim 5}. Combined diffraction images unfold their full potential when combined with an EMC algorithm \citep{Loh-2009-PRE}. This allows an orientation and averaging of diffraction images, thus enabling a high-resolution 3D reconstruction of the sample while making use of the asymmetric detector geometry. Finally, the pioneered X-ray pump--X-ray probe technique can be and already has been extended to various other experiments such as \cite{Lehmann-2016-PRA,Kimberg-2016-FD,Al-Haddad-2017-unpublished,Ferguson-2016-SciAdv,Picon-2016-NatComm}, allowing entirely new insights previously unreachable.
%
%
%