\chapter{Summary and outlook}\label{ch:summary_outlook}
For the present study, a novel FEL accelerator based X-ray pump -- X-ray probe technique was pioneered and the new soft X-ray end-station LAMP was build and commissioned to investigate X-ray induced dynamics in rare-gas clusters. Using this new instrumentation, rare-gas cluster are investigated using coincident coherent diffractive imaging (CDI) and time-of-flight mass spectroscopy. Diffraction pattern wit thus far unprecedented resolution in single particle imaging are measured by combining the single-shot diffraction pattern from multiple detectors, thus increasing the numeric aperture and extending the dynamic range of the detector system. The coincident imaging and spectroscopy pump--probe technique allows the study of single events and thus enables insights into the X-ray induced dynamics of nanometer sized rare-gas cluster. The ultrafast pump--probe technique with time delays on the few ten to hundred femtosecond timescale empowers to study the nanoplasma transition that is driving radiation damage in CDI.\\
Single xenon rare-gas cluster were illuminated by the X-ray pump beam from LCLS to induce the nanoplasma transition and at a later time delay probed by ultrafast pulses from LCLS to create a snapshot of the transition. The study of diffraction pattern shows an expansion of the cluster's electron density of XXX \% raidus over 800 fs. This is comparable to other studies using optical pump--probe methods \citep{Gorkhover-2016-NatPho}. As the relaxation processes in larger atoms are on the few femtosecond timescale, the total absorbed energy is the key driver of the nanoplasma expansion in larger clusters. As the cluster become smaller, the X-ray pump -- X-ray probe technique reveals a to optical methods hidden resonance in Xe-atoms and small cluster ($\sim <5 nm$ radius) that may be attributed to relaxation pathways only accessed after core-electron ionization \citep{Ho-2014-PRL}.\\
To prevent the nanoplasma transition from damaging the sample object (Xe-cluster), a helium layer is placed around the sample. These heterogeneous cluster consisting of a He-shell and a Xe-core are investigated using the same methods. The study shows that multiple Xe-cluster arrange in a plum-pudding type arrangement. The analysis of diffraction pattern shows that the He-droplet undergoes the nanoplasma transition, thus exhibits radiation damage, while the Xe-cluster within the droplet show no radiation damage over a pump--probe delay of 800 fs. Simultaneously, the ion-spectroscopy shows a significant increase in helium high-charge states and momentum compared to pristine He-droplets. Additionally, only few Xe-ions are detected in the final state snapshot of the time-of-flight spectrometer. The sacrificial tampered He-layer thus supplies electrons to the Xe-ions and transports kinetic energy away from the sample.\\
These results extend the insights gained from \citep{Hoener-2008-JPB} and show that tampered layer may increase the ultimate achievable resolution in single-particle imaging, where it is foreseen that radiation damage will be a limiting factor \citep{Aquila-2015-StrucDyn}. To make use of tampered layers in bio-molecule CDI, aerosol sample-injection techniques may be adapted. Until radiation damage becomes a limiting factor, a diffraction image from a single particle may be assembled from multiple detectors thus increasing the resolution drastically. From multiple detectors combined diffraction image unfold their full potential when combined with EMC algorithms \citep{Loh-2009-PRE} that allow the orientation and averaging of diffraction images thus enabling a high-resolution 3D reconstruction of the sample. 3D reconstructions of bio-molecules that may allow tomographic insight, due to the penetration depth of X-rays into soft X-rays, could be advantageous to understand the function of bio-molecules and other samples. Finally, the pioneered X-ray pump -- X-ray probe technique may be (and has been) extended to various other experimental ideas such as \citep{Picon-2016-NatComm,Kimberg-2016-FD,MacDonald-2016-RSI,Al-Haddad-2017-unpublished,Ferguson-2016-SciAdv}, allowing insights in thus far unreachable regimes.
%
%
%