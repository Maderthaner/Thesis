\section*{Abstract}\label{ch:abstract}
With the advent of lightsources of the fourth generation, such as the X-ray free electron laser (XFEL) the Linac Coherent Light Source (LCLS), structural studies on non-repetitive and non-reproducible nanoparticles such as single biomolecules appear imminent. A key challenge to increasing the resolution of single particle imaging is radiation damage. Only a sound understanding of the underlying damage processes, such as the nanoplasma phase transition, can overcome this challenge and will ultimately lead to higher resolutions. Sacrificial tamper layers have been proposed to slow sample damage but this has not been shown experimentally in aerosol nanoparticles.\\[1\baselineskip]
This work investigates the nanoplasma transition in superfluid He-, bulk Xe- and mixed HeXe-clusters using an ultrafast, XFEL accelerator-based X-ray pump--X-ray probe technique. A first X-ray pulse triggers the nanoplasma transition in the sample at intensities of \SI{\sim e17}{\watt\per\square\centi\meter}, and a second X-ray pulse probes the system at a later time, $\Delta t$, at power densities of \SI{\sim e18}{\watt\per\square\centi\meter}. The delay $\Delta t$ is varied from \SIrange{0}{800}{\femto\second} delay. The probe creates a diffraction image of the \SIrange{\sim 30}{\sim 1000}{\nano\meter} sized objects and coincidentally ions are captured via time-of-flight (TOF) spectroscopy. He- and Xe-clusters are generated through a supersonic gas expansion into a vacuum. Mixed HeXe-clusters are created when He-droplets pick up Xe-atoms in a doping unit.\\[1\baselineskip]
The study reveals single-shot snapshots of Xe-, He- and HeXe-clusters that undergo the nanoplasma phase transition. The reconstruction of a \SI{\sim 25}{\nano\meter} radius Xe-cluster with \SI{\sim 6}{\nano\meter} is shown. A large-scale structural analysis of the size and number of scatterers in Xe-clusters show a rapidly expanding particle, thus structural damage. Expansion speeds and electron temperatures of the superheated nanoplasma are derived and compared to IR pump--X-ray probe studies. TOF spectroscopy reveals a time delay $\Delta t$ dependent ionization pattern in Xe-atoms and HeXe-clusters, while pristine He-droplets do not exhibit this dependency. The structural analysis of few hundred nanometer radius HeXe-clusters show that the doped Xe-atoms arrange in small clusters of a plum-pudding type. The He-droplet acts thereby as a shell and the Xe-clusters as a sample. The X-ray pump--X-ray probe study on HeXe-clusters shows structural damage in the He-droplet, while Xe-clusters do not show structural damage 800 fs after the pump-pulse. Sample damage in pristine He-droplets is found and compared to Xe- and HeXe-clusters.
%
%
%
%