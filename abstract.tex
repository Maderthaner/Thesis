\section*{Abstract}\label{ch:abstract}
With the advent of lightsources of the fourth generation, such as the X-ray free electron laser (XFEL) the Linac Coherent Light Source (LCLS), structural studies on non-repetitive and non-reproducible nanoparticles such as single biomolecules appear imminent. A key challenge to increasing the resolution of single particle imaging is radiation damage. Only a sound understanding of the underlying damage processes, such as the nanoplasma formation, can overcome this challenge and will ultimately lead to higher resolutions. Sacrificial tamper layers have been proposed to slow X-ray sample damage but this has not been shown experimentally using nanoparticles and diffractive imaging.\\[0.5\baselineskip]
%
This thesis investigates the nanoplasma formation and expansion in superfluid He-, bulk Xe- and mixed HeXe-clusters using an ultrafast, XFEL accelerator-based X-ray pump--X-ray probe technique. A first X-ray pulse triggers the nanoplasma evolution in the sample at intensities of \SI{\sim e17}{\watt\per\square\centi\meter}, and a second X-ray pulse probes the system at a later time, $\Delta t$, at power densities of \SI{\sim e18}{\watt\per\square\centi\meter}. The delay $\Delta t$ is varied between \SIrange{0}{800}{\femto\second}. The probe-pulse creates a diffraction image of the \SIrange{\sim 30}{\sim 1000}{\nano\meter} sized objects and coincidentally ions are captured via time-of-flight (TOF) mass spectroscopy. He- and Xe-clusters are generated through a supersonic gas expansion into a vacuum. Mixed HeXe-clusters are created when He-droplets pick up Xe-atoms in a doping unit. The experiment was performed at the AMO instrument at the LCLS using the LAMP end-station. The LAMP end-station was partly designed, built, and operated as part of this work.\\[0.5\baselineskip]
%
The study reveals single-shot images of Xe-, He- and HeXe-clusters that undergo the nanoplasma formation. The reconstruction of a \SI{\sim 25}{\nano\meter} radius Xe-cluster with \SI{\sim 6}{\nano\meter} resolution is shown. A large-scale structural analysis of the size and number of scatterers in Xe-clusters show rapidly expanding nanoparticles. Expansion speeds and electron temperatures of the nanoplasma are derived. Reconstructions and 2D computer models of few-hundred nanometer radius HeXe-clusters shows that the doped Xe-atoms arrange in small clusters of a plum-pudding type. The He-droplet acts thereby as a sacrificial layer and the Xe-clusters as sample. TOF spectroscopy reveals charge transfer ionization in HeXe-cluster that reduce the electronic damage of Xe-clusters by sacrificing the He-tamper layers. Diffraction images and 2D computer models of HeXe-clusters shows structural damage in the He-droplet, while the Xe-clusters do not exhibit structural damage up to \SI{800}{\femto\second} after the pump-pulse. As a measure of the effectiveness of the sacrificial layer, radiation damage of pristine He- and Xe-clusters are compared to HeXe-clusters.
%
%
%
%