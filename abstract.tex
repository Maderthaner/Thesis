\section*{Abstract}\label{ch:abstract}
With the advent of lightsources of the fourth generation, such as the \textit{X-ray free electron laser} the Linac Coherent Light Source (LCLS), structural studies on non-repetitive and non-reproducible nanoparticles such as single biomolecules appear imminent. A key challenge to increasing the resolution of single particle imaging is radiation damage. Only a sound understanding of the underlying damage processes, such as the \textit{nanoplasma formation}, can overcome this challenge and will lead to higher resolutions. Recently, \textit{sacrificial layers} have been proposed to slow the \textit{nanoplasma expansion}.\\[0.4\baselineskip]
%
This thesis investigates nanoplasma formation and expansion in superfluid He-, bulk Xe- and mixed HeXe-clusters, whereby HeXe-cluster are a testbed nanoparticle for sacrificial layers. The clusters are generated through a supersonic gas expansion into a vacuum and investigated using an ultrafast X-ray pump--X-ray probe technique from the LCLS. A first X-ray pulse triggers the nanoplasma evolution at intensities of \SI{\sim e17}{\watt\per\square\centi\meter}, and a second X-ray probe-pulse creates diffraction images of the nanoplasma at a later time, $\Delta t$, with power densities of \SI{\sim e18}{\watt\per\square\centi\meter}. $\Delta t$ is varied between \SIrange{0}{800}{\femto\second} and furthermore ion-spectra are captured via time-of-flight (TOF) mass spectroscopy. The experiment was performed at the AMO instrument using the LAMP end-station. The LAMP end-station was partly designed, built, and operated as part of this thesis work.\\[0.4\baselineskip]
%
The study reveals single-shot diffraction images and TOF spectra of Xe-, He- and HeXe-clusters with an unprecedented resolution, for example, the reconstruction of a \SI{\sim 25}{\nano\meter} radius Xe-cluster with \SI{\sim 6}{\nano\meter} resolution is shown. The diffraction images and corresponding reconstructions show the nanoplasma formation and expansion of nanoparticles. An analysis of several hundred diffraction images shows that the initial Xe-cluster radius increases by \SI{\sim20}{\percent} over \SI{800}{\femto\second}. Hereby, the expansion-speed is \SI{\sim 15250}{\meter\per\second} and the electron temperature is \SI{\sim125}{\electronvolt}. Reconstructions and 2D computer models of few-hundred nanometer radius HeXe-clusters shows that the doped Xe-atoms arrange in small clusters of a \textit{plum-pudding type}. TOF spectra show that Xe-cluster transfer kinetic energy to the sacrificial layer and that the He-layer supplies electrons to the ionized Xe-clusters. Diffraction images and 2D computer models of HeXe-clusters show that Xe-cluster do not exhibit a Nanoplasma expansion in the initial \SI{800}{\femto\second}, while the He-tamper-layer undergoes a nanoplasma expansion. As a measure of the effectiveness of the sacrificial layer, the radiation damage of pristine He- and Xe-clusters, as well as HeXe-clusters are compared to each other.
%
%
%
%