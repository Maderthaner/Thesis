\chapter{Introduction}
Investigating matter with light is one of the most fundamental approaches to study nature. Whether we look at things by eye or use more advanced methods for example using lasers, it is an invaluable tool to understand nature. Through light, we can study the shapes of objects, investigate fundamental particles, study quantum mechanics and so much more. Key aspect in almost any study involving light is it's wavelength. The wavelength of light can make you feel cozy at home with a modern LED light bulb or shroud your home in a blue, uncomfortable haze when you use older light bulbs. At the end of the 19th century, R\"ontgen discovered the then novel X-radiation and quickly realized its impact due to its different wavelength. Soon, he was able to take first medical X-ray images, later X-rays could be used to investigate fundamental aspects of atoms. The success story continues until today, where short wavelength X-rays enables the study of the workhorse in human bodies, namely proteins. The shape of a protein, which is only a few nanometers in size, cannot be seen in a microscope anymore and only X-rays can be used to unravel their shape with sufficient resolution. The shape of a biomolecules is of particular interest because it defines its biological function. The chemical structure of most human biomolecules are fairly similar and affect the function little. Medical drugs often aim to affect biological functions such that the specific shape of a protein becomes very important for drug research and drug design.\\
However, there is more to light than just its wavelength. The intensity of light plays a crucial role, for example when we are sitting in front of a fire and the warm glaze of infrared (IR) radiation can be felt on our skin. Anyway, using high intense radiation allows us to create an environment comparable to the inside of the sun, where fusion processes create energy and other interesting high energy processes are happening. Intense light is also needed if you want to study the very small because the interaction between matter and light becomes very small on length scales of for example a protein. To study the shape of a protein a certain interaction is needed and since we do not want to change the protein, we have to adjust the intensity of light.\\
Last but not least, light can come in short flashes for example when taking a photograph. At night, when the flash is too long, the pictures often get washed out. It is similar when you study nature, if you want to look at some movement or more general dynamics, your light flash must be short enough to resolve this dynamic, otherwise it gets washed out. Proteins function also through movement. Many of their dynamics could already be deciphered but it is unclear what one would discover if one would look at even faster timescales.\\
Summarizing, a light flash can be described by by its wavelength, intensity and duration. To study matter, e.g. proteins, one has to optimize these three parameters a lot. TTherefore, a new kind of light source was developed that is particularly interesting for imaging the very small on the nanometer scale. The first one of these light sources, a so called free electron laser (FEL), was build in Hamburg, Germany. First experiments with FELs that showed proof of principle studies showcasing capabilities of FELs as a highly intense and ultra short light pulse source. As a result, more advanced X-ray free electron laser (XFEL) were built to further improve capabilities. The first hard XFEL was built in Menlo Park, California at Stanford University and is called the Linac Coherent Light Source (LCLS). It is a ~4.1 km long machine\footnote{Measurement on Google maps from the injector building to the far experimental hall.}, the most straightest building in the world\footnote{The European XFEL in Hamburg, Germany is currently build to similar engineering requirements.}, is build underneath the ground and the interstate I-280 crosses it via a bridge\footnote{On a fun note, the accelerator part of LCLS is actually older than the I-280 and do avoid interruptions of the experiments, the bridge was build long before the intersate and there are pictures of this bridge that is not connected to any streets.}. This massive machine delivers what is needed to use imaging techniques that have been out of reach so far. To name a few, with LCLS it is possible to study protein crystals of much smaller size, one can study single particles for example viruses and cells. And LCLS is not limited to imaging techniques, one can study atoms and molecules in thus far unreachable regimes and also understand magnetism better. The reason these new areas of study area accessible through LCLS lay in the light parameters. LCLS produces light with a very short wavelength (X-rays from 4.6 nm to 0.1 nm), it has a very high intensity ($10^18 Ws/cm^{2}$) and ultra short pulses ($1-500 fs$). The combination of these parameters is unique in the world and enables us to look at things that have been hidden so far.\\
Let us now imagine that we expose a protein to the very intense X-rays from LCLS (or any other XFEL). Within the first moment of interaction from the light and the proteins, the protein scatters photons distinctive to it shape that we are able measure and hence recover the shape of the protein. But the protein will also absorb the light and therefore energy and will become very hot, much like the conditions on the surface of the sun. The hot protein will expand quickly and disintegrate into its atomic components. While the scattering process is desired the inevitable absorption process will ultimately hinder the scattering process and limit the resolution of our measurement. The damage that occurs to the sample is an ultra-fast process that can only be studied with an even faster light pulse. It is a process that thus far is known to exist but has been hidden due to the limitation of other light sources. XFEL offer now the opportunity to study this process that is applicable to a variety of X-ray imaging techniques ranging from crystallography to spectroscopic applications over to the imaging of single particles.\\
This thesis discusses an experiment performed with LCLS to study the effects and dynamics induced by radiation damage on homogenous xenon and helium clusters and heterogeneous clusters consistent of xenon and helium. A detailed and time-resolved X-ray pump -- X-ray probe study is undertaken to first induce X-ray related dynamics via a X-ray pump pulse and then create a single-shot image of the cluster at a given time delay. We combine the single particle imaging technique with a time-of-flight mass spectrometer that yields insights into ionization dynamics, absorbed energies and particle heterogeneity. Rare gas clusters are used as sample target as they are an ideal nanosamples in the gas phase. They are easy to produce and easy to transport to the interaction region with LCLS. Furthermore they can be tuned in size and multiple rare gases, here xenon and helium, can be combined to form a single nanometer sized heterogeneous cluster. We present the data from the study in form of real space images from clusters complemented by diffraction images. Furthermore, we correlate the time-of-flight data to each image.\\
The thesis is organized as follows, chapter \ref{ch:fundamental_concepts} discusses fundamental aspects and background information to the performed studies. Chapter \ref{ch:exp_setup} focuses on the experimental setup of this study, chapter \ref{ch:results} presents the results obtained and finally \ref{ch:summary_outlook} sumarises the previous chapters and provides an outlook for further studies.