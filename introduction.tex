\chapter{Introduction}
%Investigating matter with light is one of the most fundamental approaches to study nature. Whether we look at things by eye or use more advanced methods for example using lasers, it is an invaluable tool to understand nature. Through light, we can study the shapes of objects, investigate fundamental particles, study quantum mechanics and so much more. Key aspect in almost any study involving light is it's wavelength. The wavelength of light can make you feel cozy at home with a modern LED light bulb or shroud your home in a blue, uncomfortable haze when you use older light bulbs. At the end of the 19th century, R\"ontgen discovered the then novel X-radiation and quickly realized its impact due to its different wavelength. Soon, he was able to take first medical X-ray images, later X-rays could be used to investigate fundamental aspects of atoms. The success story continues until today, where short wavelength X-rays enables the study of the workhorse in human bodies, namely proteins. The shape of a protein, which is only a few nanometers in size, cannot be seen in a microscope anymore and only X-rays can be used to unravel their shape with sufficient resolution. The shape of a biomolecules is of particular interest because it defines its biological function. The chemical structure of most human biomolecules are fairly similar and affect the function little. Medical drugs often aim to affect biological functions such that the specific shape of a protein becomes very important for drug research and drug design. \\
% R\"ontgen discovered the X-radiation \citep{Roentgen-NP} and quickly realized its impact due to its novel attributes. Soon, he was able to take the first medical X-ray image.
Following the discovery of X-radiation at the end of the 19$^{\text{th}}$ century by R\"ontgen, X-rays have been used to understand and investigate matter in a unprecedented way. Soon after this discovery, X-rays were used to take the first medical X-ray image. Later, X-rays lead to the understanding of fundamental aspects of atoms \citep{Siegbahn-NP} and crystals \citep{Laue-NP,Bragg-NP} and today, the success story continues in various fields of science. A large and active scientific field is structural biology. Here, X-rays are being used to study the shape of proteins through crystallography. Proteins are the so-called ``workhorse'' in the human body and the shape of a protein defines its biological function. Unfortunately not all proteins can be grown into large crystals making the molecular shape determination challenging. The advent of new lightsources, namely the X-ray free electron laser (XFEL) \citep{Ackermann-2007-NPho}, enable new methods to study bio-molecules and other particles with intense and short-pulse X-rays \citep{Chapman-2006-NatPhys,Chapman-2011-Nature}. However, all matter irradiated by intense X-ray radiation disintegrates into its atomic components on ultrafast timescales\footnote{Ultrafast timescales are on the order of atto- to nanoseconds.} \citep{Neutze-2000-Nature}. This process is a form of \textit{radiation damage} and is a major challenge for imaging methods with XFEL \citep{Aquila-2015-StrucDyn}.\\[1\baselineskip]
%
The underlying principle of these imaging methods is that intense X-ray pulses diffract from a single macromolecular structure before they destroy it. In diffraction imaging, images of single, nanometer-sized particles can be taken for structural research. First experiments have successfully delivered single-shot diffraction images of biological particles, such as viruses \citep{Seibert-2011-Nature} and non-biological particles, such as rare-gas clusters \citep{Gomez-2014-Science}. It has been a rapidly developing field that has recently succeeded in visualizing 3D images of nano-objects \citep{Ekeberg-2015-PRL,Barke-2015-NatComm}.\\[1\baselineskip]
%
A key driver of this success is the development of new lightsources with more and more spectral brightness. The first hard XFEL was built at Stanford University and is called the Linac Coherent Light Source (LCLS) \citep{Emma-2010-NatPho}. It is a $\sim 4.1$ km long machine\footnote{Measurement on Google maps from the injector building to the far experimental hall.} that produces X-rays with wavelengths from 4.6 nm to 0.1 nm, has a highly intense beam of up to $10^{18} \tfrac{\text{W s}}{\text{cm}^{2}}$ and ultra-short pulses ranging from $1-500$ fs. LCLS is a user facility and scientists from around the world may apply to use this lightsource for their experiments. The Atomic, Molecular and Optical (AMO) physics instrument was the first of seven instruments in operation and is a focal point for experiments ranging from biological imaging to basic science \citep{Bostedt-2016-RMP}.\\[1\baselineskip]
%
Let us now imagine that we expose a nanoparticle to the highly intense X-rays from LCLS. Within the first moment of light-nanoparticle interaction, photons elastically scatter in an angular distribution distinctive to the shape of the particle that we are able measure in 2D. However, the particle will also absorb the rays from the first moment on, which leads to inner atomic-shell vacancies \citep{Young-2010-Nature}. Subsequent processes, such as the Auger-decay, occur only a few femtoseconds later and eventually the nanoparticle is transformed into a nanoplasma. The nanoplasma is very hot, comparable to the conditions of the surface of the sun and several forces will expand the plasma on the femtosecond timescale \citep{Gorkhover-2016-NatPho}. Finally, the nanoplasma disintegrates into its atomic components. The absorption process, which triggers this cascade of events in the nanoparticle, is unavoidable when the particle is exposed to intense X-rays and so is radiation damage in the sample.\\[1\baselineskip]
%
The principle to ``outrun'' the radiation induced damage with femtosecond lightpulses is challenging for two reasons. One, it limits the intensity output of the lightsource, and two, sample damages due to photoionization and changing scattering factors are inevitable. Ultimately, these reasons will limit the achievable resolution in single particle imaging. But radiation damage can also be addressed in other ways. Computer models can account for known processes \citep{Quiney-2010-NatPhys} and novel methods from AMO physics, such as molecule alignment \citep{Kupper-2014-PRL} and sacrificial tamper layer \citep{Hau-Riege-2004-PRE,Hau-Riege-2010-PRL} may inhibit it.\\[1\baselineskip]
%
This thesis describes the nanoplasma phase transition in detail. A novel X-ray pump -- X-ray probe method was employed to study the nanoplasma evolution in the testbed nanoparticles superfluid He-, bulk Xe- and mixed HeXe-cluster. The pump-pulse initiates a cascade of X-ray induced effects in these particles and the probe-pulse creates a ``snapshot'' at a later time delay. Coincident diffraction imaging and time-of-flight mass spectroscopy data is taken to answer the following questions:
\begin{itemize}
	\item How does X-ray induced damage compare to optical light induced damage?
	\item On what timescales do absorption cross-sections change due to X-ray irradiation?
	\item How do mixed HeXe-clusters self-organize as nanoparticles?
	\item Does He compare to sacrificial tamper layers in mixed HeXe-clusters?
\end{itemize}
%
This document is organized as follows. Chapter \ref{ch:fundamental_concepts} discusses fundamental aspects that are being used throughout this study; chapter \ref{ch:exp_setup} describes the experimental setup at the AMO instrument at LCLS - describing in particular the LAMP end-station and pnCCD detectors; chapter \ref{ch:methods} discusses several computational methods. Chapter \ref{ch:results} presents the results of the pump--probe study; and chapter \ref{ch:summary_outlook} summarizes the present work and provides an outlook for further studies.
%
%
%