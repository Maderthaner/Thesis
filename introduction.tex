\chapter{Introduction}
%Investigating matter with light is one of the most fundamental approaches to study nature. Whether we look at things by eye or use more advanced methods for example using lasers, it is an invaluable tool to understand nature. Through light, we can study the shapes of objects, investigate fundamental particles, study quantum mechanics and so much more. Key aspect in almost any study involving light is it's wavelength. The wavelength of light can make you feel cozy at home with a modern LED light bulb or shroud your home in a blue, uncomfortable haze when you use older light bulbs. At the end of the 19th century, R\"ontgen discovered the then novel X-radiation and quickly realized its impact due to its different wavelength. Soon, he was able to take first medical X-ray images, later X-rays could be used to investigate fundamental aspects of atoms. The success story continues until today, where short wavelength X-rays enables the study of the workhorse in human bodies, namely proteins. The shape of a protein, which is only a few nanometers in size, cannot be seen in a microscope anymore and only X-rays can be used to unravel their shape with sufficient resolution. The shape of a biomolecules is of particular interest because it defines its biological function. The chemical structure of most human biomolecules are fairly similar and affect the function little. Medical drugs often aim to affect biological functions such that the specific shape of a protein becomes very important for drug research and drug design. \\
% R\"ontgen discovered the X-radiation \citep{Roentgen-NP} and quickly realized its impact due to its novel attributes. Soon, he was able to take the first medical X-ray image.
Following the discovery of \textit{X-radiation} at the end of the 19$^{\text{th}}$ century by R\"ontgen, X-rays have been used to understand and investigate matter in unprecedented ways. Soon after this discovery, X-rays were used to take the first medical image. Later, X-rays lead to the understanding of fundamental aspects of atoms \citep{Siegbahn-NP} and crystals \citep{Laue-NP,Bragg-NP}. Today, the success story continues in various fields of science. A large and active scientific field is structural biology. Here, X-rays are being used to study the structure of proteins through crystallography. Proteins are the so-called ``workhorse'' in the human body and the structure of a protein defines its biological function. Unfortunately not all proteins can be grown into large crystals making the molecular structure determination challenging. However, new lightsources, namely \textit{X-ray free electron lasers} (XFELs) \citep{Ackermann-2007-NPho} have more and more peak \textit{brightness} allowing the structure determination of smaller and smaller protein-crystals \citep{Chapman-2011-Nature}.\\[1\baselineskip]
%
The first hard XFEL was built at Stanford University and is called the Linac Coherent Light Source (LCLS) \citep{Emma-2010-NatPho}. It is a multikilometer long machine that produces X-rays with wavelengths from \SIrange{4.6}{0.1}{\nano\meter}, has intensities of up to \SI{\sim e18}{\watt\per\square\centi\meter}, and ultra-short pulses ranging from \SIrange{1}{500}{\femto\second}. Scientists from various disciplines can apply to use this lightsource for their experiments that are conducted in several instruments. The Atomic, Molecular, and Optical (AMO) physics instrument at LCLS was the first of seven instruments in operation and is a focal point for experiments ranging from biological imaging to basic science \citep{Bostedt-2016-RMP}.\\[1\baselineskip]
%
The beam parameters that are available at the AMO instrument opened up an entirely new method for structural biologists, which is the \textit{single particle imaging} (SPI). With SPI, the shape of single bio-molecules can be determined through \textit{diffractive imaging} \citep{Chapman-2006-NatPhys}. First experiments have successfully delivered single-shot diffraction images of biological particles, such as viruses \citep{Seibert-2011-Nature} and non-biological particles, such as rare-gas clusters \citep{Gomez-2014-Science}. It has been a rapidly developing field that has recently succeeded in visualizing 3D images of nano-objects \citep{Ekeberg-2015-PRL,Barke-2015-NatComm}.\\[1\baselineskip]
%
But the highly-intense pulses that enable diffractive imaging lead to unusual questions. 
%All matter irradiated by intense X-ray radiation disintegrates into its atomic components on ultrafast\footnote{Ultrafast timescales are on the order of atto- to nanoseconds.} timescales \citep{Neutze-2000-Nature}. This process is a form of \textit{radiation damage} and is a major challenge for imaging methods with XFEL \citep{Aquila-2015-StrucDyn}.
%So, the underlying principle of diffractive imaging is that intense X-ray pulses diffract from a single macromolecular structure before they destroy it.\\[1\baselineskip]
%
%Imagine now that we expose a nanoparticle to the highly intense X-rays from LCLS. Within the first moment of light-nanoparticle interaction, photons elastically scatter in an angular distribution distinctive to the shape of the particle that we are able measure in 2D. However, the particle will also absorb the rays from the first moment on, which leads to inner atomic-shell vacancies \citep{Young-2010-Nature}. Subsequent processes, such as the \textit{Auger-decay}, occur only a few femtoseconds later and eventually the nanoparticle is transformed into a \textit{nanoplasma}. Several forces will expand the plasma on a femtosecond timescale \citep{Gorkhover-2016-NatPho}. Finally, the nanoplasma disintegrates into its atomic components. The absorption process, which triggers this cascade of events in the nanoparticle, is unavoidable when the particle is exposed to intense X-rays and so is radiation damage in the sample.\\[1\baselineskip]
When an intense soft X-ray pulse irradiates a nanoparticle, the particle will simultaneously absorb and diffract X-rays with the absorption cross-section being much larger than the scattering cross-section. From the first moment of light-matter interaction, the absorption will lead to inner atomic-shell vacancies \citep{Young-2010-Nature} and these vacancies make the above cross-sections time-dependent. Subsequent ionization cascades, such as the \textit{Auger-decay}, occur only a few femtoseconds later and the nanoparticle is thus transformed into a nanoplasma on the femtosecond timescale. Several forces will expand the plasma \citep{Gorkhover-2016-NatPho} until eventually, the nanoplasma disintegrates into its atomic components.\\[1\baselineskip]
%
The underlying principle of diffractive imaging, which is that intense X-ray pulses diffract from a single macromolecular structure before they destroy it, is now challenging since sample damages occur while the pulse propagates through the nanoparticle. Changing scattering factors and trapped, delocalized electrons will affect the diffraction image. While diffractive imaging is still feasible, these reasons will limit the achievable resolution in single particle imaging. Several ideas have been proposed to address radiation damage. Computer models can account for known processes \citep{Quiney-2010-NatPhys}; molecule alignment prior imaging provides additional information \citep{Kupper-2014-PRL}; and sacrificial tamper layers can compensate for radiation damage processes \citep{Hau-Riege-2004-PRE,Hau-Riege-2010-PRL}. While experimental data exists for the first two ideas, sacrificial tamper layers that are placed around aerosol injected particles have until now not been studied experimentally and are in focus of this work.\\[1\baselineskip]
%The principle of ``outrunning'' the radiation induced damage with femtosecond lightpulses is challenging for two reasons. One, it limits the intensity output of the lightsource, and two, sample damages due to photoionization and changing scattering factors are inevitable. 
%Ultimately, these reasons will limit the achievable resolution in single-particle imaging. But radiation damage can also be addressed in other ways. Computer models can account for known processes \citep{Quiney-2010-NatPhys} and novel methods from AMO physics, such as molecule alignment \citep{Kupper-2014-PRL} and sacrificial tamper layers \citep{Hau-Riege-2004-PRE,Hau-Riege-2010-PRL} may inhibit it.\\[1\baselineskip]
%
\begin{figure}
	\centering
		\includegraphics[width=0.80\textwidth]{images/tamper-layer.png}
	\caption[Computer simulations of \SI{7.5}{\nano\meter} radius aluminum spheres with tamper layers]{Computer simulations of \SI{7.5}{\nano\meter} radius aluminum spheres that are illuminated by intense soft X-rays with a fluence of \SI{2.5e8}{\joule\per\square\centi\meter}. On the left is a pristine Al-sphere and on the right is a Al-sphere with a \SI{2.5}{\nano\meter} thick silicon tamper layer. From \citep{Hau-Riege-2010-PRL}. Reprinted with permission from APS.}
	\label{fig:tamper-layer}
\end{figure}
%
Sacrificial tamper layers supply electrons to the photoionized sample and transport kinetic energy away from the sample. This delays the nanoplasma expansion of the sample (see Figure \ref{fig:tamper-layer}. An effective tamper layer would ideally be thin and uniform around the sample to minimize its background signal. A first idea of a tamper layer is water, since bio-molecules typically have a water-based layer when injected into the imaging setup via an aerosol gas jet. However, water-based layers become disordered in thicker layers and are usually uneven around macro-molecules \citep{Aquila-2015-StrucDyn}. An alternative are helium-based layers, where, for example, a sample particle is placed inside a helium-droplet.\\[1\baselineskip]
%
This thesis discusses experimental data of sacrificial tamper layers, which are in form of a helium-droplet, that has been placed around xenon-clusters, which represent the aerosol sample. The data is complemented by the study of pristine Xe- and He-clusters. A novel \textit{X-ray pump--X-ray probe} method was employed to study the nanoplasma formation in those samples. Here, the pump-pulse triggers the nanoplasma formation and the probe-pulse creates a ``snapshot'' at a later time. Coincident diffraction imaging and time-of-flight mass spectroscopy data is measured and further analyzed with phase-retrieval algorithms to yield real-space images of single, aerosol nanoparticles with thus far unprecedented resolution. This analysis answers the following questions:
\begin{itemize}
	%\item How does X-ray induced damage compare to optical light induced damage?
	%\item On what timescales do absorption cross-sections change due to X-ray irradiation?
	\item How does an X-ray induced nanoplasma-transition affect Xe-clusters?
	%\item On what timescales are diffraction images affected at current resolution?
	\item How do mixed HeXe-clusters self-organize as nanoparticles?
	\item Does helium compare to sacrificial tamper layers in mixed HeXe-clusters?
\end{itemize}
%
This document is organized as follows. Chapter \ref{ch:fundamental_concepts} discusses fundamental aspects that are used throughout this study; Chapter \ref{ch:exp_setup} describes the experimental setup at the AMO instrument at LCLS and in particular the LAMP end-station with it's detectors; Chapter \ref{ch:methods} discusses several computational methods. Chapter \ref{ch:results} presents the results of the pump--probe study; and Chapter \ref{ch:summary_outlook} summarizes the present work and provides an outlook for further studies.
%
%
%