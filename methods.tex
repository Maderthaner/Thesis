\chapter{Methods}\label{ch:methods}
\section{The computing environment at LCLS}
- Include basics around the PSANA interface\\
- For example how the date is converted, then stored and\\
- the analysis opportunities along the way
- I think this will be a longer subsection since a lot of my work went into this and I'm regularly contacted about it.
- Short introduction what we have to go through\\
- Reminder of detectors and analysis environment
\subsection{PSANA - Python interaction with LCLS computing}
%%%
\section{pnCCD photon detectors}
- Describe signal on the pnCCDs\\
- Calibrations and corrections - use LAMP paper\\
- single hits\\
- multiple hits
\subsection{Signal analysis}
- Present data from 1500eV photon energy on Xe backfilled chamber with the pnCCDs in spectroscopy mode to argue that the pedestal and offset corrections are enough to correct for fluoresence.\\
- Masked areas in image
%%%
\section{Combining multiple pnCCD detectors}
- Explain how I combined pnCCD detectors to perform reconstructions on it.\\
- Can reuse material from the LAMP pnCCD paper
%%%
\section{Hitfinding}
- Discuss the hitfinding.\\
- iTOF vs. pnCCD\\
- vs. Actual dynamics visible in diffraction images
%%%
\section{Phase retrieval from a single diffraction pattern}
- Short intro into phase retrieval
\subsection{Solving the inverse problem}
\subsection{2D reconstructions and limitations}
\subsubsection{Hawk program}
- Describe Filipe's program
\subsubsection{Resolution enhancement through combination of rear and front pnCCD}
- Showcase difference of rear pnCCD only vs. front + rear pnCCD vs. front + rear pnCCD 'cropped' for best results. Recycle work from LAMP pnCCD paper
\subsection{1D reconstructions}
- Describe my algorithm in 1D in detail
%%%
%\section{Signal on time of flight detector}
%- Explain what we see in Xe, He and XeHe sample environments.\\
%- and further analysis aspects, e.g. Aqiris / psana interface.
%%%
\section{Summary of methods}
- Comprehensive summary 